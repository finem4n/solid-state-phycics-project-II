\documentclass{article}

\usepackage[utf8]{inputenc}
\usepackage[MeX]{polski}
\usepackage{graphicx}   %do rysunków
\usepackage{wrapfig}    %do rysunków otoczonych tekstem
\usepackage{color}      %do użycia podst. kolorów oraz zdefiniowanych kolorów 
%do kolorowych referencji do rysunków, cytowań:
\usepackage{multirow}
\usepackage{longtable}
\usepackage{float}
\usepackage{indentfirst}
\usepackage[shortlabels]{enumitem}
\usepackage[colorlinks=true,linkcolor=blue,citecolor=green]{hyperref}

%do kolorowych referencji do rysunków, cytowań:
\usepackage{multicol}
\usepackage{colortbl}


%do pdfow
\usepackage{pdfpages}

\textwidth=16cm
\textheight=23cm
\topmargin=-2cm
\oddsidemargin=0cm

\title{}
\author{ }
\date{}

\begin{document}

\begin{table}[h!]
\centering
\begin{tabular}{|p{2.1cm}|p{2.1cm}|p{2.1cm}|p{2.1cm}|p{2.1cm}|p{2.1cm}|}\hline
\multicolumn{2}{|c|}{\textbf{Wstęp do fizyki ciała stałego}}
& \multicolumn{4}{c|}{\textbf{Projekt 2, zestaw 1}} \\ \hline
\multicolumn{2}{|l|}{Konrad Marciniak} & e-mail:
& \multicolumn{3}{c|}{konrad.marciniak.stud@pw.edu.pl}
\\ \hline
data:& & nr indeksu: & 311730 & grupa: & W2 \\ \hline
\multicolumn{6}{|c|}{\begin{tabular}[c]{@{}c@{}}Oświadczam, że jestem jedynym autorem/jedyną
autorką niniejszego projektu.\\Jestem świadomy/świadoma odpowiedzialności w przypadku podania fałszywej informacji.\\ \\ (podpis studenta)\end{tabular}} \\ \hline
\end{tabular}
\end{table}

\section*{Zadanie 1 - fonony}

\section*{Zadanie 2 - nanostruktury półprzewodnikowe}

\section*{Zadanie 3 - analiza termiczna}
\end{document}
